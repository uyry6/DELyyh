\documentclass[12pt,a4paper]{article}
\input{latexmacros.tex}

\title{Finite Automata and Regular Expresions \\in Haskell}
\author{Yuanho Yao}
\date{\today}
\hypersetup{pdfauthor={Yuanho Yao}, pdftitle={Finite Automata and Regular Expresions in Haskell}}

\begin{document}

\maketitle

\begin{abstract}
  This article investigates how finite automata and regular expressions are connected. We give an implementation of DFA(deterministic finite automata) and NFA(non-deterministic finite automata) and RE(regular expression) in Haskell.
  We first encode DFA and then give a code converting RE to NFA and NFA to DFA respectively.
  This program enables us to verify if a string is accepted in certain DFA. And we can determine if a string is an instance of certain RE.
\end{abstract}

\vfill

\tableofcontents

\clearpage

\input{lib/Mycode.lhs}

\addcontentsline{toc}{section}{Bibliography}
\bibliographystyle{plain}
\bibliography{references.bib}

\end{document}
